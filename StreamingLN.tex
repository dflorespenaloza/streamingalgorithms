\documentclass[11pt]{article}
\usepackage{geometry}                % See geometry.pdf to learn the layout options. There are lots.
\geometry{letterpaper}                   % ... or a4paper or a5paper or ... 
%\geometry{landscape}                % Activate for for rotated page geometry
%\usepackage[parfill]{parskip}    % Activate to begin paragraphs with an empty line rather than an indent
\usepackage{graphicx}
\usepackage{amssymb}
\usepackage{amsmath}
\usepackage{epstopdf}
%\DeclareGraphicsRule{.tif}{png}{.png}{`convert #1 `dirname #1`/`basename #1 .tif`.png}

\newtheorem{theorem}{Theorem}
\newtheorem{lemma}{Lemma}
\newtheorem{corollary}{Corollary}
\newtheorem{claim}{Claim}
\newtheorem{remark}{Remark}
\newtheorem{question}{Question}
\newtheorem{definition}{Definition}


\title{Streaming Algorithms}
\author{The Authors}
%\date{}                                           % Activate to display a given date or no date

\begin{document}
\maketitle


%%---------------------------------------------------------------------
\section{Rewarding Subway's users}

\subsection{The problem}

\subsection{What is Big Data?}

\subsection{The streaming model}

\subsection{A simple algorithm}

\subsection{Exercises}


%%---------------------------------------------------------------------
\section{Heavy hitters {\small [Manku and Motwani 2002]}}


\subsection{The problem}

\subsection{A warm up: find the majority element}


\subsection{A heavy hitters algorithm}


\subsection{Exercises}


%%---------------------------------------------------------------------

\section{Length of the stream with few space {\small [Morris 1978, Flajolet 1985]}}

\subsection{Definition of the problem}

\subsection{Applications}

\subsection{A simple implementation (not in the streaming model)}

\subsection{Morris's algorithm} $\,$

\subsection{Some ideas to take away from Flajolet's analysis}

\subsection{Exercises}


%%---------------------------------------------------------------------
\section{Number of distinct elements}

\subsection{The problem}

\begin{definition}[The number of distinct elements problem]
The input is an stream $S$ with $m$ elements, each taken from $\{1, \hdots, n\}$.
After reading the stream once, we want to know the number of distinct elements in $S$.
\end{definition}

This problem is a particular case of the so-called \emph{frequency moments of} $S$:


\begin{definition}[Frequency moments of a stream]
Let $S = s_1, \hdots, s_m$ be a string with $m$ elements, each taken from $\{1, \hdots, n\}$.
Let $m_i$ be the number of apparences of $i$ in $S$, namely, $m_i = | \{j : s_j = i\} |$.
For every integer $k \geq 0$, the \emph{$k$-th frequency moment of $S$} is

$$F_k = \Sigma^n_{i = n} m_i^k.$$ 
\end{definition}


Thus $F_0$ is the number of distinct elements of $S$ while $F_1$ is its length.
The second frequency momento $F_2$ is called the \emph{surprise index}
and gives interesting information of a stream that is useful in several statistical applications.
The next lecture presents the surprise index more in detail.

\subsection{An algorithm}

The following simple and nice algorithm from Alon et al. estimates the number of distinct elements of a stream.
It does so by estimating the number of bits needed to encode the elements in the string.

The algorithm first picks a finite field large enough where all distinct values of the stream can be
represented. Then, it picks at random a hash function (a polynomial) that maps each elemento to
a member in the field. 
For each element in the stream, the algorithm applies the hash function and then
looks at the number consecutive 0's at the right of the binary representation.
That number is what the algorithm is interested in, it returns the largest number it sees.

Here it is important to remark that once a hash function is chosen, the algorithm is deterministic. In the analysis of 
the algorithm (following section) it is important that for every element of the string $s_i$,
$z_i$ (the value $s_i$ is mapped to) is uniformly distributed over ${0, ..., 2^d-1}$.
Something that also is important about the hash function is that it is pairwise independent,
a notion that is explained in the next section.




\begin{verbatim}
d = smallest integer so that 2^d > n
a,b = two random values in {0, ..., 2^d-1} chosen uniformly and independently
r = 0

while(there are elements in the stream)
   s_i = next in the stream
   z_i = a s_i + b (product and addition are computed in the field GF(2^d))
   r(z_i) = largest j so that the j rightmost bits of z_i are all 0
   r = max{r(z_i), r}       
endwhile

return r

\end{verbatim}



\subsection{What is the algorithm computing?}




\subsection{Not a formal analysis but so ideas to take away}




\subsection{Exercises}


\begin{enumerate}

\item Give an algorithm that solves the distinct elements problem in $O(|S|)$ time
and $O(n)$ memory. Prove its correctness.

\item Run Alon et al.' algorithm with $S = 1 2 3 3 5 5 1 2 5$ and $a=2$ and $b=3$.
Show all calculations in every while loop iteration.

\item In you favorite programing language, write Alon et al.'s algorithm. 
Run 10 experiments for $n=10$ and $N=100$ with random streams.
How far is the algorithm from the solution?  
\end{enumerate}



\section{The surprise index of a stream}



\begin{thebibliography}{XXX}

\bibitem{Ref1} Reference 1.

\end{thebibliography}

\end{document}  