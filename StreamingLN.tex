\documentclass[11pt]{article}
\usepackage{geometry}                % See geometry.pdf to learn the layout options. There are lots.
\geometry{letterpaper}                   % ... or a4paper or a5paper or ... 
%\geometry{landscape}                % Activate for for rotated page geometry
%\usepackage[parfill]{parskip}    % Activate to begin paragraphs with an empty line rather than an indent
\usepackage{graphicx}
\usepackage{amssymb}
\usepackage{amsmath}
\usepackage{epstopdf}
%\DeclareGraphicsRule{.tif}{png}{.png}{`convert #1 `dirname #1`/`basename #1 .tif`.png}

\newtheorem{theorem}{Theorem}
\newtheorem{lemma}{Lemma}
\newtheorem{corollary}{Corollary}
\newtheorem{claim}{Claim}
\newtheorem{remark}{Remark}
\newtheorem{question}{Question}
\newtheorem{definition}{Definition}


\title{Streaming Algorithms}
\author{The Authors}
%\date{}                                           % Activate to display a given date or no date

\begin{document}
\maketitle


%%---------------------------------------------------------------------
\section{Rewarding Subway's users}

\subsection{The problem}

\subsection{What is Big Data?}

\subsection{The streaming model}

\subsection{A simple algorithm}

\subsection{Exercises}


%%---------------------------------------------------------------------
\section{Heavy hitters {\small [Manku and Motwani 2002]}}


\subsection{The problem}

\subsection{A warm up: find the majority element}


\subsection{A heavy hitters algorithm}


\subsection{Exercises}


%%---------------------------------------------------------------------

\section{Length of the stream with few space {\small [Morris 1978, Flajolet 1985]}}

\subsection{Definition of the problem}

\subsection{Applications}

\subsection{A simple implementation (not in the streaming model)}

\subsection{Morris's algorithm} $\,$

\subsection{Some ideas to take away from Flajolet's analysis}

\subsection{Exercises}


%%---------------------------------------------------------------------
\section{Number of distinct elements}

\subsection{The problem}

\begin{definition}[The number of distinct elements problem]
The input is an stream $S$ with $m$ elements, each taken from $\{1, \hdots, n\}$.
After reading the stream once, we want to know the number of distinct elements in $S$.
\end{definition}

This problem is a particular case of the so-called \emph{frequency moments of} $S$:


\begin{definition}[Frequency moments of a stream]
Let $S = s_1, \hdots, s_m$ be a string with $m$ elements, each taken from $\{1, \hdots, n\}$.
Let $m_i$ be the number of apparences of $i$ in $S$, namely, $m_i = | \{j : s_j = i\} |$.
For every integer $k \geq 0$, the \emph{$k$-th frequency moment of $S$} is

$$F_k = \Sigma^n_{i = n} m_i^k.$$ 
\end{definition}






Of course, the problem is easy if there is no restriction on the total memory.
Insist that the total memory must be polylog in $n$.

Introduce $F_k$ and say that the problem is to compute $F_0$.



\subsection{A simple solution}
If there is no restriction on the memory an algorithm can use, the problem can be easily solved:






\subsection{An algorithm}

Alon et al.'s solution.

\subsection{What is the algorithm computing?}

The order of magnitude interpretation (with a table).


\subsection{Not a formal analysis but so ideas to take away}


\subsection{Excersises}



\begin{thebibliography}{XXX}

\bibitem{Ref1} Reference 1.

\end{thebibliography}

\end{document}  